\chapter{Einleitung und Problemstellung}

Die Pfadplanung ist ein fundamentales Problem in vielen Bereichen wie Computerspiele \cite{Krishnaswamy.2009}, Robotik \cite{latombe}, Navigation und Logistik \cite{Botea.2011}. Statische und dynamische Umgebungen stellen unterschiedliche Herausforderungen dar. Weiter gibt es Szenarien, in denen mehrere Akteure gleichzeitig Pfade finden müssen. Um diese Probleme anzugehen, gibt es verschiedene Ansätze, die in dieser Arbeit diskutiert werden. Dafür existieren unterschiedliche Kategorien von Algorithmen. Diese Arbeit definiert die Problemstellung, gibt eine Einführung in die Algorithmen zur Lösung des Problems und führt verschiedene Anwendungsszenarien auf. Es wird grundlegendes Verständnis für die Funktionsprinzipien fundamentaler Pfadplanungsalgorithmen aufgebaut und Ansätze für die umgebungsbedingte Optimierung erläutert. Dazu wird die Effizienz der verschiedenen Pfadplanungsprinzipien beleuchtet.

%todo PAthfinding Begriff auch einführen


%todo PAthfinding Begriff auch einführen

Mithilfe dieser Arbeit ist es möglich ein Grundverständnis für Funktionsprinzipien und Anwendungsszenarien der Pfadplanungsalgorithmen zu erlangen. Diese kann ausgebaut und zur praktischen
Umsetzung oder weiteren Forschung eingesetzt werden kann.

%todo • definiert / erläutert das Thema
%todo • motiviert die Thematik (Aktualität / Relevanz)
%todo • stellt die Bedeutung der Arbeit heraus
%todo • liefert interessante/relevante Hintergrundinformationen
%todo • umfassende Literaturrecherche
%todo • bindet Thema in einem weiter gefassten Kontext ein
%todo • bzw. grenzt es sinnvoll zu anderen Arbeiten ab
%todo • formuliert (am Ende) eine Hypothese / Fragestellung
