\chapter{Einleitung und Problemstellung}


Das Finden solcher kürzester Pfade ist das Ziel der Algorithmen, die in dieser Arbeit diskutiert werden.

Spielen Eingenzung eventuell

DIRECTION ORIENTED PATHFINDING IN VIDEO GAMES
Current solutions for pathfinding, in the context of video games, either provide a high speed search by sacrificing accuracy or produce an optimal path but using more time and resources [1]. How to find an optimal path more efficiently is still an area of study.


Pathfinding is a fundamental component ofmany important applications in the fields of GPS [1], video games [2], robotics [3], logistics, and crowd simulation [4, 5]and canbe implemented in static, dynamic, and real-time environments. Although anumberofdevelopmentshaveimprovedthe accuracy and efficiency of pathfinding techniques over the past two decades, the problem still attracts a great deal of research. Currently, the most important area concerns the provision of high-performance, realistic paths for users. In general, there are different variations of the pathfinding problem [6, 7], such as single-agent pathfinding search, multiagent pathfinding search, adversarial pathfinding, dynamic changes in the environment, heterogeneous terrain, mobile units, and incomplete information. Each of these problems has different applications in different fields. Generally, pathfinding consists oftwo main steps: graph generation and a pathfinding algorithm.
The graph generation problem for “terrain topology” is considered a foundation of robotics and video games applications.



