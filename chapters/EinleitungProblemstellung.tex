\pagestyle{plain}
\chapter{Einleitung und Problemstellung}

In einem Zeitalter, in dem Smartphones, Videospiele und Navigationsanwendungen Massenprodukte geworden sind, steigt auch der Bedarf an effizienteren Lösungen bei der Pfadplanung. Pfadplanung ist die Suche nach Wegen in unterschiedlichen Umgebungen. Daraus ergibt sich die Frage nach den Funktionsprinzipien und Anwendungsgebieten verschiedener Lösungsansätze. Dazu ist es von Interesse, welche Strategien sich in der Anwendung bewährt haben. 

Diese Ausarbeitung definiert die Herausforderung der Pfadplanung und gibt eine Einführung in die Algorithmen zur Lösung des Problems. Es wird ein grundlegendes Verständnis für die Funktionsprinzipien von Pfadplanungsalgorithmen aufgebaut und Optimierungskonzepte erläutert. Darüber hinaus wird die Effizienz der verschiedenen Pfadplanungsansätze beleuchtet und ein Überblick über Algorithmen gegeben, die in der Praxis Anwendung finden. Die Arbeit basiert auf den Originalquellen der wichtigsten Pfadplanungsalgorithmen und gibt die Ergebnisse aktueller Untersuchungen wieder.
 

