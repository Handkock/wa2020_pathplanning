\chapter{Einleitung und Problemstellung}

In einem Zeitalter, in dem Smartphones, Videospiele und autonome Fahrzeuge Massenprodukte geworden sind, steigt auch der Bedarf an effizienteren Lösungen bei der Pfadplanung. Pfadplanung ist die Suche nach Wegen in unterschiedlichen Umgebungen. Daraus ergibt sich die Frage nach den Funktionsprinzipien und Anwendungsgebieten verschiedener Lösungsansätze. Dazu ist es von Interesse, welche Strategien sich in der Anwendung bewährt haben. 

Diese Arbeit definiert die Herausforderung der Pfadplanung und gibt eine Einführung in die Algorithmen zur Lösung des Problems. Es wird ein grundlegendes Verständnis für die Funktionsprinzipien von Pfadplanungsalgorithmen aufgebaut und Optimierungskonzepte erläutert. Darüber hinaus wird die Effizienz der verschiedenen Pfadplanungsansätze beleuchtet und ein Überblick über Algorithmen gegeben, die in der Praxis Anwendung finden. Die Arbeit basiert auf den Originalquellen der wichtigsten Pfadplanungsalgorithmen und gibt die Ergebnisse aktueller Untersuchungen wieder.
 


%todo • stellt die Bedeutung der Arbeit heraus
 %• Diese Arbeit erklärt/stellt dar/gibt Einblick/...
%Mithilfe dieser Arbeit ist es möglich ein Grundverständnis für Funktionsprinzipien und Anwendungsszenarien der Pfadplanungsalgorithmen zu erlangen. Diese kann ausgebaut und zur praktischen Umsetzung oder weiteren Forschung eingesetzt werden kann.%



%Die Suche nach dem kürzesten Pfad stellt in vielen Anwendungsbereichen ein Problem dar. Die Algorithmen zur Lösung des Problems sind Suchalgorithmen, die in Graphen arbeiten. Die meisten von ihnen lösen das \textit{single-source shortest path} Problem, eine spezielle Variante des Problems des kürzesten Pfades. Hierbei wird nach dem schnellsten Weg von einem Startknoten zu allen anderen Knoten in einem Graphen gesucht. Uninformierte Pfadplanungsalgorithmen arbeiten blind und sind in jeder Umgebung anwendbar. Informierte Algorithmen hingegen arbeiten mit heuristischen Informationen, und schätzen den Weg zum Ziel mithilfe verschiedener Metriken. Der informierte Algorithmus A* arbeitet auf Grundlage des Dijkstra-Algorithmus, und bietet eine verbesserte Laufzeit. Die hierarchische Optimierung des A* Algorithmus, HPA*, arbeitet mit einem abstrahierten Suchraum, und läuft durchschnittlich zehnmal schneller als A*. Deshalb wird HPA* viel in Videospielen genutzt, wo Pfadplanungsprobleme meist in Echtzeit gelöst werden müssen. Die gleichzeitige Suche nach mehreren, kollisionsfreien Pfaden, stellt eine besondere Herausforderung dar, die mithilfe des Konzepts von Multi-Agent Pathfinding gelöst wird.%

