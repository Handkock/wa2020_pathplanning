\chapter{Das Problem des kürzesten Pfades}

In diesem Abschnitt wird ein Einstieg in das Problem des kürzesten Pfades (engl. Shortest Path Problem, SSP) gegeben. Ein optimaler kürzester Pfad besitzt eine möglichst kurze Distanz von Start zum Ziel\cite{Madkour.2017}. Viele Pfadplanungsverfahren repräsentieren das Problem des kürzesten Pfades als Suchproblem in einem Graphen.


\section{Begriffsdefinitionen}

Ein Graph ist eine mathematische Darstellung von Objekten, die aus Punkten und Verbindungen zwischen den Punkten bestehen. Ein Graph ist ein Tupel zwei disjunkter Mengen: $G_{def}= (V,E)$. Die Elemente $v \in V$ heißen Knoten (vertices) und die Elemente $e \in E$ sind die Kanten (edges). Die Graphen können gewichetet oder ungewichtet sein. Bei gewichteten Graphen jeder Kante ist eine Gewichtung zugeordnet. Diese Gewichtungen wirken sich auf Pfade durch den Graphen aus\cite{Gross.2004}. Pfad in einem Graphen ist ein Tupel $\left ( x_{0}, x_{1}, x_{2}, ..., x_{n} \right )$ mit $x_{i} \in V$.  Im Hinblick auf Problem des kürzesten Pfades wird ein gewichteter Graphen mit einem Startknoten $s \in V$ und den Zielknoten $d \in V$ betrachtet. Ziel ist: \textit{den Pfad  $\left ( s, ..., d \right )$ mit dem geringsten Gewicht zu finden}\cite{Madkour.2017}. 

\section{Arten des Shortest Path Problems}

Das Problem des kürzesten Pfads besteht in verschiedenen Varianten. Die in dieser Arbeit diskutierten Algorithmen lösen das Problem des \textbf{single-source shortest path} (SSSP). Sie finden den kürzesten Weg von einem festen Startpunkt zu allen Knoten eines Graphen \cite{Gu.2018}. Das \textbf{single-destination shortest path} Problem (SDSP)  beschreibt die Suche nach den kürzesten Pfaden aller Knoten zu einem festen Zielpunkt. Des Weiteren existiert das \textbf{single-pair shortest path} Problem (SPSP), welches den kürzesten Weg zwischen genau einem Start- und Zielpunkt darstellt. Es unterscheidet sich nur durch das Abbruchkriterium vom SSSP Problem \cite{Ottmann.2017}. Im worst-case muss die Distanz vom Startknoten zu allen anderen Knoten berechnet werden, um den kürzesten Pfad zum Zielknoten zu finden. Weitere Algorithmen lösen das \textbf{all-pairs shortest path Problem} (APSP). Sie finden den kürzesten Pfad zwischen allen Knotenpaaren eines Graphen \cite{Cormen.2009}.
