\chapter{Das Problem des kürzesten Pfades}

In diesem Abschnitt wird ein Einstieg in das Problem des kürzesten Pfades (engl. Shortest Path Problem, SSP) gegeben. Ein optimaler kürzester Pfad besitzt eine möglichst kurze Distanz von Start zu Ziel\cite{Madkour.2017}. Viele Pfadplanungsverfahren repräsentieren das Problem des kürzesten Pfades als Suchproblem in einem Graphen[?A Comprehensive Study on Pathfinding Techniques for Robotics and Video Games]. %todo umformulieren, weil keine Quelle vorhanden, vlt. viele Pfadplanungsverfahren lassen sich einfach auf das graph zurückzuführen

\section{Begriffsdefinitionen}

%todo Graph matematisch beschreiben, eventuell beim Schmitz nachgucken
%ein Pfad bestimmen, und s und d konsisten nutzen
Ein Graph ist eine mathematische Darstellung von Objekten, die aus Punkten und Verbindungslinien bestehen. Ein Graph \textit{G} ist ein Tupel zwei disjunkter Mengen \textit{V} und \textit{E}. Die Elemente der Menge V werden als Knoten und die Elemente der Menge E werden als Kanten bezeichnet. Einer Kante können ein oder zwei Knoten als Endpunkte zugewiesen werden. Kanten können Gewichtungen haben. Diese Gewichtungen wirken sich auf Pfade durch den Graphen aus\cite{Gross.2004}. Im Zuge eines Pfadplanungsverfahrens beinhaltet die Menge \textit{V} einen Startknoten \textit{s}, einen Endknoten \textit{d} und eine Menge aus gewichteten Kanten \textit{E}. %todo besser mathematisch, d,e el E
Ziel des Verfahrens ist es den Pfad von \textit{s} nach \textit{d} mit dem geringsten Gewicht zu finden\cite{Madkour.2017}. 

%Zur Lösung von Pfadplanungsproblemen definieren wir einen Startknoten $s$ und einen Endknoten $d$, wobei $s, d \in V$, sowie eine Menge gewichteter Kanten $E$. Den kürzeste Pfad definieren wir als $SP = min(d \rightarrow s)$ Ziel des Verfahrens ist es den kürzesten Pfad zu finden.


\section{Arten von Shortest Path Problem}