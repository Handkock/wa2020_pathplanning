\chapter{Anwendungen}

1. Any-Angle ?
2. Multi-client ?

\section{Anwendung von A* in Videospielen}

Der A* Algorithmus wird in vielen Videospielen für die Bewegungskontrolle von Nicht-Spieler Charakteren (engl. non-player character, NPC) genutzt. Der Umgebung des Spiels liegen in vielen Fällen festen Knoten oder ein Kachelsystem zugrunde. Basiert die Bewegungssteuerung auf dem System der Umgebung, werden die Bewegungen der NPCs jedoch häufig vorhersehbar [1].

In Abbildung 1 ist ein Spiel zu sehen, in dem der Computer den kürzesten Pfad in einem kachelbasierten Spielfinden finden soll. Die Algorithmen A*, Dijkstra und BFS werden in ihren Eigenschaften verglichen. Dijkstra und BFS durchsuchen beide das gesamte Spielfeld, während A* knapp die Hälfte der Kacheln untersucht. Dennoch schafft es Dijkstras Algorithmus schneller einen idealen Pfad zu berechnen als A* [2].

In komplexeren Levels (siehe Abbildung 2 und 3) schafft es A* jedoch schneller als Dijkstra und BFS einen idealen Pfad zu berechnen. Grundsätzlich eignet sich der A* Algorithmus also besser für die Bestimmung des kürzesten Pfads in Videospielen als Dijkstras Algorithmus oder die Breitensuche. [2]

[1][Pathfinding in partially explored games environments]
[2][Comparative Analysis of Pathfinding Algorithms A *, Dijkstra, and BFS on Maze Runner Game]

[Abb 1][Maze Runner]
[Abb 2][Maze Runner2]
[Abb 3][Maze Runner3]