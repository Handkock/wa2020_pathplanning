\chapter{Fazit}

In dieser Arbeit wurde die Problemstellung des kürzesten Pfads dargestellt. Es wurden die grundlegenden Funktionsprinzipien von Algorithmen zur Pfadplanung erklärt. Der Unterschied zwischen uninformierten und informierten Algorithmen wurde herausgearbeitet und verschiedene Metriken für Schätzverfahren genannt. Zusätzlich wurden die besonderen Herausforderungen verschiedener Anwendungsgebiete diskutiert und Ansätze der umgebungsspezifischen Optimierung erläutert. Abschließend wurde der Ansatz von Multi-Agent Pathfinding-Algorithmen zur parallelen Suche mehrerer, kollisionsfreier Pfade gegeben.