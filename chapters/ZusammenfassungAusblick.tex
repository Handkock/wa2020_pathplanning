\chapter{Fazit}

In dieser Arbeit wurde die Problemstellung des kürzesten Pfades dargestellt und in den Kontext der Graphentheorie gesetzt. Es wurden die grundlegenden Funktionsprinzipien der wichtigsten Algorithmen zur Pfadplanung, wie Dijkstras Algorithmus und A*, erklärt. Anhand dieser wurde der Unterschied zwischen uninformierten und informierten Algorithmen herausgearbeitet. Zu erkennen ist, dass informierte Suchalgorithmen aufgrund ihrer heuristischen Schätzverfahren eine verbesserte Laufzeit gegenüber den uninformierten Algorithmen bieten. Zusätzlich wurden die besonderen Herausforderungen von Videospielen diskutiert. Trotz der Vorteile gegenüber dem Dijkstra-Algorithmus kann A* weiter verbessert werden. Dazu wurde der HPA*-Algorithmus als eine Erweiterung von A* vorgestellt. Abschließend wurde der Ansatz von Multi-Agent Pathfinding-Algorithmen zur Suche mehrerer, kollisionsfreier Pfade beleuchtet.


