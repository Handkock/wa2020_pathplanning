\chapter{Fazit}

In dieser Arbeit wurde die Problemstellung des kürzesten Pfades dargestellt. Es wurden die grundlegenden Funktionsprinzipien von Algorithmen zur Pfadplanung erklärt. Der Unterschied zwischen uninformierten und informierten Algorithmen wurde herausgearbeitet und verschiedene Metriken für Schätzverfahren genannt. Der A* Algorithmus bietet eine verbesserte Laufzeit gegenüber der Breitensuche und dem Dijkstra Algorithmus, was anhand einer Studie belegt wurde. Zusätzlich wurden die besonderen Herausforderungen verschiedener Anwendungsgebiete diskutiert und die Funktionsprinzipien der hierarchischen Optimierung des A*-Algorithmus und deren Anwendung beschrieben. Abschließend wurde der Ansatz von Multi-Agent Pathfinding-Algorithmen zur parallelen Suche mehrerer, kollisionsfreier Pfade gegeben.

%Die Suche nach dem kürzesten Pfad stellt in vielen Anwendungsbereichen ein Problem dar. Die Algorithmen zur Lösung des Problems sind Suchalgorithmen, die in Graphen arbeiten. Die meisten von ihnen lösen das \textit{single-source shortest path} Problem, eine spezielle Variante des Problems des kürzesten Pfades. Hierbei wird nach dem schnellsten Weg von einem Startknoten zu allen anderen Knoten in einem Graphen gesucht. Uninformierte Pfadplanungsalgorithmen arbeiten blind und sind in jeder Umgebung anwendbar. Informierte Algorithmen hingegen arbeiten mit heuristischen Informationen, und schätzen den Weg zum Ziel mithilfe verschiedener Metriken. Der informierte Algorithmus A* arbeitet auf Grundlage des Dijkstra Algorithmus, und bietet eine verbesserte Laufzeit. Die hierarchische Optimierung des A* Algorithmus, HPA*, arbeitet mit einem abstrahierten Suchraum, und läuft durchschnittlich zehnmal schneller als A*. Deshalb wird HPA* viel in Videospielen genutzt, wo Pfadplanungsprobleme meist in Echtzeit gelöst werden müssen. Die gleichzeitige Suche nach mehreren, kollisionsfreien Pfaden, stellt eine besondere Herausforderung dar, die mithilfe des Konzepts von Multi-Agent Pathfinding gelöst wird.%