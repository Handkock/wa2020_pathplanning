\chapter{Fazit}

In dieser Arbeit wurde die Problemstellung des kürzesten Pfades dargestellt und in den Kontext der Graphentheorie gesetzt. Es wurden die grundlegenden Funktionsprinzipien der wichtigsten Algorithmen zur Pfadplanung erklärt. Anhand dieser wurde der Unterschied zwischen uninformierten und informierten Algorithmen herausgearbeitet. Uninformierte Pfadplanungsalgorithmen arbeiten blind und liefern immer die optimale Lösung. Informierte Suchalgorithmen hingegen bieten aufgrund ihrer heuristischen Schätzverfahren eine verbesserte Laufzeit. Der informierte Algorithmus \textit{A*} findet im Allgemeinen schneller eine Lösung als der \textit{Dijkstra-Algorithmus}. Dennoch kann der Algorithmus weiter verbessert werden. Der HPA*-Algorithmus ist eine Erweiterung von \textit{A*}. Er läuft ca. zehnmal schneller als \textit{A*} und findet daher viel Anwendung in Videospielen, wo Pfade häufig in Echtzeit berechnet werden müssen. Abschließend wurden die \textit{Multi-Agent Pathfinding}-Algorithmen WHCA*, FAR und \textit{Bounded Multi-Agent A*} zur Suche mehrerer, kollisionsfreier Pfade beleuchtet.




%Die Suche nach dem kürzesten Pfad stellt in vielen Anwendungsbereichen ein Problem dar. Die Algorithmen zur Lösung des Problems sind Suchalgorithmen, die in Graphen arbeiten. Die meisten von ihnen lösen das \textit{single-source shortest path} Problem, eine spezielle Variante des Problems des kürzesten Pfades. Hierbei wird nach dem schnellsten Weg von einem Startknoten zu allen anderen Knoten in einem Graphen gesucht. Uninformierte Pfadplanungsalgorithmen arbeiten blind und sind in jeder Umgebung anwendbar. Informierte Algorithmen hingegen arbeiten mit heuristischen Informationen, und schätzen den Weg zum Ziel mithilfe verschiedener Metriken. Der informierte Algorithmus A* arbeitet auf Grundlage des Dijkstra-Algorithmus, und bietet eine verbesserte Laufzeit. Die hierarchische Optimierung des A* Algorithmus, HPA*, arbeitet mit einem abstrahierten Suchraum, und läuft durchschnittlich zehnmal schneller als A*. Deshalb wird HPA* viel in Videospielen genutzt, wo Pfadplanungsprobleme meist in Echtzeit gelöst werden müssen. Die gleichzeitige Suche nach mehreren, kollisionsfreien Pfaden, stellt eine besondere Herausforderung dar, die mithilfe des Konzepts von Multi-Agent Pathfinding gelöst wird.%
