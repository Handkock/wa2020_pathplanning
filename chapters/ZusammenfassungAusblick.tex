\chapter{Fazit}

In dieser Arbeit wurde die Problemstellung des kürzesten Pfades dargestellt. Es wurden die grundlegenden Funktionsprinzipien von Algorithmen zur Pfadplanung erklärt. Der Unterschied zwischen uninformierten und informierten Algorithmen wurde herausgearbeitet und verschiedene Metriken für Schätzverfahren genannt. Zusätzlich wurden die besonderen Herausforderungen verschiedener Anwendungsgebiete diskutiert und die Funktionsprinzipien der hierarchischen Optimierung des A*-Algorithmus und deren Anwendung beschrieben. Abschließend wurde der Ansatz von Multi-Agent Pathfinding-Algorithmen zur parallelen Suche mehrerer, kollisionsfreier Pfade gegeben.