%%%%%%%%%%%%%%%%%%% vorlage.tex %%%%%%%%%%%%%%%%%%%%%%%%%%%%%
%
% LaTeX-Vorlage zur Erstellung von Projekt-Dokumentationen
% im Fachbereich Informatik der Hochschule Trier
%
% Basis: Vorlage svmono des Springer Verlags
%
%%%%%%%%%%%%%%%%%%%%%%%%%%%%%%%%%%%%%%%%%%%%%%%%%%%%%%%%%%%%%

\documentclass[envcountsame,envcountchap, deutsch]{i-studis}

\usepackage{makeidx}         	% Index
\usepackage{multicol}        	% Zweispaltiger Index
%\usepackage[bottom]{footmisc}	% Erzeugung von Fu�noten

%%-----------------------------------------------------
%\newif\ifpdf
%\ifx\pdfoutput\undefined
%\pdffalse
%\else
%\pdfoutput=1
%\pdftrue
%\fi
%%--------------------------------------------------------
%\ifpdf
\usepackage[pdftex]{graphicx}
\usepackage{epstopdf}
\usepackage[pdftex,plainpages=false]{hyperref}
%\else
%\usepackage{graphicx}
%\usepackage[plainpages=false]{hyperref}
%\fi

%%-----------------------------------------------------
\usepackage{color}				% Farbverwaltung
%\usepackage{ngerman} 			% Neue deutsche Rechtsschreibung
\usepackage[english, ngerman]{babel}
\usepackage[utf8]{inputenc} 	% Erm�glicht Umlaute-Darstellung
%\usepackage[utf8]{inputenc}  	% Erm�glicht Umlaute-Darstellung unter Linux (je nach verwendetem Format)

%-----------------------------------------------------
\usepackage{listings} 			% Code-Darstellung
\lstset
{
	basicstyle=\scriptsize, 	% print whole listing small
	keywordstyle=\color{blue}\bfseries,
								% underlined bold black keywords
	identifierstyle=, 			% nothing happens
	commentstyle=\color{red}, 	% white comments
	stringstyle=\ttfamily, 		% typewriter type for strings
	showstringspaces=false, 	% no special string spaces
	framexleftmargin=7mm, 
	tabsize=3,
	showtabs=false,
	frame=single, 
	rulesepcolor=\color{blue},
	numbers=left,
	linewidth=146mm,
	xleftmargin=8mm
}
\usepackage{textcomp} 			% Celsius-Darstellung
\usepackage{amssymb,amsfonts,amstext,amsmath}	% Mathematische Symbole
\usepackage[german, ruled, vlined]{algorithm2e}
\usepackage[a4paper]{geometry} % Andere Formatierung
\usepackage{bibgerm}
\usepackage{array}
\usepackage{graphicx}
\usepackage{tabularx}
%\usepackage{tabularx}
\hyphenation{Ele-men-tar-ob-jek-te  ab-ge-tas-tet Aus-wer-tung House-holder-Matrix Le-ast-Squa-res-Al-go-ri-th-men} 		% Weitere Silbentrennung bei Bedarf angeben
\setlength{\textheight}{1.1\textheight}
\pagestyle{myheadings} 			% Erzeugt selbstdefinierte Kopfzeile
\makeindex 						% Index-Erstellung

%--------------------------------------------------------------------------
\begin{document}
%------------------------- Titelblatt -------------------------------------
\title{Funktionsprinzipien und Anwendungen von Algorithmen zur Pfadplanung}
\project{Ausarbeitung zur Vorlesung Wissenschaftliches Arbeiten}
%--------------------------------------------------------------------------
\supervisor{Titel Vorname Name} 		% Betreuer der Arbeit
\author{Bearbeiter 1: Fredrik Kappmeier \\Bearbeiter 2: Vladislav Paskar \\Bearbeiter 3: Valentin Thum}							% Autor der Arbeit
\groupid{146}
\address{Trier,} 							% Im Zusammenhang mit dem Datum wird hinter dem Ort ein Komma angegeben
\submitdate{21.06.2020} 				% Abgabedatum
%\begingroup
%  \renewcommand{\thepage}{title}
%  \mytitlepage
%  \newpage
%\endgroup
\begingroup
  \renewcommand{\thepage}{Titel}
  \mytitlepage
  \newpage
\endgroup
%--------------------------------------------------------------------------
\frontmatter 
%--------------------------------------------------------------------------
\kurzfassung

Die folgende Arbeit beschäftigt sich mit der Pfadplanung. Diese hat das Ziel %todo komma ?
den kürzesten Pfad zwischen Start- und Endpunkt zu finden. Die Breiten- und Tiefensuche sind grundlegende Lösungsansätze, die eine optimale Lösung garantieren. Sie gehören zu den uninformierten Algorithmen, die blind arbeiten. Informierte Algorithmen arbeiten mit heuristischen Informationen, um die Suche zu verkürzen. Dijkstras Algorithmus ist einer der wichtigsten Algorithmen in der Pfadplanung. Er stellt die Grundlage für den in der Anwendung weit verbreiteten A* Algorithmus dar. Zur Anpassung an spezifische Umgebungseigenschaften existieren diverse Erweiterungen von A*, wie hierarchisches A* Pathfinding. Multi-Agent Pathfinding-Algorithmen planen für mehrere Akteure kollisionsfreie kürzeste Pfade.
 			% Kurzfassung Deutsch/English
\tableofcontents 						% Inhaltsverzeichnis
%--------------------------------------------------------------------------
\mainmatter                        		% Hauptteil (ab hier arab. Seitenzahlen)
%--------------------------------------------------------------------------
% Die Kapitel werden in separaten .tex-Dateien abgelegt und hier eingebunden.
\chapter{Einleitung und Problemstellung}

Die Pfadplanung ist ein fundamentales Problem in vielen Bereichen wie Computerspiele \cite{}, Robotik \cite{}, Navigation und Logistik \cite{Botea.2011}. Statische und dynamische Umgebungen stellen unterschiedliche Herausforderungen dar. Weiter gibt es Szenarien, in denen mehrere Akteure gleichzeitig Pfade finden müssen. Um diese Probleme anzugehen, gibt es verschiedene Ansätze, die in dieser Arbeit diskutiert werden. Dafür existieren unterschiedliche Kategorien von Algorithmen. Diese Arbeit definiert die Problemstellung, gibt eine Einführung in die Algorithmen zur Lösung des Problems und führt verschiedene Anwendungsszenarien auf. Es wird grundlegendes Verständnis für die Funktionsprinzipien fundamentaler Pfadplanungsalgorithmen aufgebaut und Ansätze für die umgebungsbedingte Optimierung erläutert. Dazu wird die Effizienz der verschiedenen Pfadplanungsprinzipien beleuchtet.
\chapter{Das Problem des kürzesten Pfades}

In diesem Abschnitt wird ein Einstieg in das Problem des kürzesten Pfades (englisch shortest path problem) gegeben. Ein optimaler kürzester Pfad besitzt eine möglichst kurze Distanz von Start zu Ziel\cite{Madkour.2017}. Viele Pfadplanungsverfahren repräsentieren das Problem des kürzesten Pfades als Suchproblem in einem Graphen[?A Comprehensive Study on Pathfinding Techniques for Robotics and Video Games].

\section{Begriffsdefinitionen}

Ein Graph ist eine mathematische Darstellung von Objekten die aus Punkten und Verbindungslinien bestehen. Dabei ist ein Graph G als aus zwei Mengen V und E bestehend definiert. Die Elemente der Menge V werden als Knoten und die Elemente der Menge E werden als Kanten bezeichnet. Einer Kante werden ein oder zwei Knoten als Endpunkte zugewiesen. Kanten können Gewichtungen haben. Diese Gewichtungen wirken sich auf Pfade durch den Graphen aus\cite{Gross.2004}. Im Zuge eines Pfadplanungsverfahrens beinhaltet die Menge V einen Startknoten s und einen Endknoten d und eine Menge aus gewichteten Kanten E. Ziel des Verfahrens ist es den Pfad von s nach d mit dem geringsten Gewicht zu finden\cite{Madkour.2017}. 

\section{Eine Übersicht der Pfadplanungsansätze}

Das Weltmodell kann für verschiedene Einsatzbereiche unterschiedlich dargestellt werden. Eine Abbildung der realen Welt versucht Merkmale aus dieser zu erfassen.  Folgende Ansätze für die Raumdarstellung führen jeweils Fragestellungen auf Suchprobleme in Graphen zurück.
Eine Methode zur Pfadplanung in geometrischen Karten oder Wegkartenverfahren ist die Suche per Voronoi-Diagramm. Eine anderer Ansatz ist die Suche mit einem Sichtbarkeitsgraphen.
\subsection{Das Voronoi Diagram}
 Bei Erstellung des Voronoi-Diagramms wird der Raum in Regionen (Voronoi-Regionen) zerlegt. Jede Region enhält ihr Zentrum und die Punkte, die zu diesem Zentrum euklidisch näher liegen, als zu jedem anderen Zentrum. \cite{voronoi}
\subsection{Sichtbarkeitsgraph} 
Ein Sichtbarkeitsgraph enthält alle gegenseitig sichtbaren Orte. Eine Kante zwischen zwei Knotenpunkten existiert genau dann, wenn die Knoten sich gegenseitig sehen. Dies ist der Fall, wenn es keine Hindernisse zwischen den Knoten gibt. \cite{visG1}

\subsection{Potentialfeldverfahren}
Das Potentialfeldverfahren bietet einen Ansatz, bei dem der Suchraum von Gradienten überspannt wird. Aus den Gradienten werden dann mögliche Bewegungsrichtungen ersichtlich. ?? \cite{potField}

\subsection{Zellzerlegungsverfahren}

Die Zellzerlegung ist ein bekanntes Hindernisvermeidungsverfahren, das den hindernisfreien Konfigurationsraum in eine endliche Sammlung nicht überlappender(disjunkter) konvexer Polygone, sogenannte Zellen, zerlegt. In diesen kann leicht ein Pfad gefunden werden. Obwohl dieses Verfahren rechenintensiv ist, besteht sein Vorteil gegenüber anderen Ansätzen zur Planung von Roboterpfaden darin, dass die Zellzerlegung unter geeigneten Annahmen vollständig aufgelöst ist. Andere Verfahren verwenden beispielsweise eine Roadmap oder Potentialfeldverfahren.\cite{cd}

\subsection{Topologische Graphen}
Diese Gruppe der Verfahren wird in zwei Gruppen unterteilt. Es gibt informierte und uninformierte Suchalgorithmen. Die beiden Gruppen werden im folgenden Kapitel näher besprochen.
%todo Warum Anwendung im Funktionsprinzipienteil?

\chapter{Funktionsprizipien von Pfadplanungsalgorithmen}

Pfadplanung im Hinblick auf Graphentheorie werden in informierte und uninformierte Algorithmen unterteilt. Uninformierte Pfadplanung, auch ``Blindsuche`` genannt, wird in alle Richtungen durchgeführt. In der informierten Pfadplanung, im Gegensatz, wird die Richtung zum Ziel geschätzt und in diese Richtung weitergesucht\cite{comAnal}. %todo klingt gut ?

\section{Uninformierte Algorithmen}
Algorithmen dieser Gruppe werden verwendet, wenn keine Informationen über die Entfernung vom Startknoten $s$ zum Zielknoten $d$ bekannt sind.
Sie garantieren, dass der kürzeste Pfad gefunden wird, falls ein solcher Pfad existiert\cite{comAnal}. Weil uninformierte Algorithmen über alle möglichen erreichbaren Knoten gehen, gibt es einen großen Bedarf an Speicherplatz und Leistung. Aus diesem Grund finden diese Verfahren seltenere Anwendung als die informierten \cite{sim}. Die am meisten verwendeten Algorithmen dieser Gruppe sind die Breitensuche, die Tiefensuche und der Algorithmus von Dijkstra.

\subsection{Breitensuche}

Die Breitensuche (engl. Breadth-First Search, BFS) ist ein simpler Suchalgorithmus, welcher in Graphen angewendet werden kann \cite{Cormen.2009}. Der Algorithmus kann den kürzesten Pfad zwischen zwei Punkten zu bestimmen. Dazu ist er in der Lage einen Pfad zu finden, der möglichst wenig andere Pfade kreuzt, oder bestmöglich Hindernisse vermeidet \cite{Lee.1961}. BFS dient als Grundmuster für Shortest Path Algorithmen in ungerichteten Graphen \cite{Ottmann.2017}. Die Breitensuche kann jedoch sowohl in gerichteten als auch ungerichteten Graphen genutzt werden. Sie hat eine Laufzeit von $O(V + E)$, also die Summe der Anzahl Knoten und Kanten \cite{Cormen.2009}.

Die Breitensuche durchläuft einen Graphen systematisch und schrittweise in alle Richtungen. Gegeben ist ein Graph $G = (V, E)$, mit einem festgelegten Startknoten $s$, sowie ein Inkrement $k=1$. Nun werden folgende Schritte in einer Schleife ausgeführt:
\begin{itemize}
\item[1.] Finde alle Knoten, welche $k$ Kanten von $s$ entfernt sind.
\item[2.] Rechne $k + 1$.
\end{itemize}
Diese Schritte werden so lange wiederholt, bis alle Knoten im Graphen gefunden wurden, die von $s$ erreicht werden können. Wird ein Knoten von BFS entdeckt, so wird die Distanz zum Startknoten $s$ berechnet. Anschließend wird der Knoten zu einem speziellen „Breitensuchbaum“ hinzugefügt, welcher alle vom Startknoten aus erreichbaren Knoten beinhaltet. Die Zweige des Baums stellen den kürzesten Pfad von $s$ zu den Knoten des Baumes dar \cite{Cormen.2009}.

\subsubsection{Anwendung der Breitensuche}
Die Breitensuche kann bei Pfadfindung in einem Irrgärten-Spiel verwendet werden\cite{Permana.2018}. Außerdem können BFS und ihre Hybridversionen\cite{effHyb} in der Graphendarstellung von Graphikprozessor für Graphentraversierung benutzt werden\cite{scaleGPU}. %todo umformulieren richtig, wie es in der Quelle ist

\subsection{Tiefensuche}

Die Tiefensuche (engl. Depth-First Search, DFS) stellt das Gegenstück zur Breitensuche dar. Sie untersucht eine Kette von Folgeknoten bis das Ende der Kette erreicht wurde. Grundlegend lässt sich der Algorithmus in zwei Phasen unterteilen, die „Suche nach tieferliegenden Knoten“[1] und das sogenannte „Backtracking“[2] \cite{Tarjan.1972}:
\begin{itemize}
	\item[1] DFS untersucht den zuletzt gefundenen Knoten auf ausgehende Kanten. Existiert eine unentdeckte Kante, folgt DFS der Kante zum nächsten Knoten. Existiert keine weitere Kante, wechselt der Algorithmus in Phase 2.
	\item[2] DFS setzt zum überliegenden Knoten zurück. Dieser wird auf eine unentdeckte Kante untersucht. Existiert eine weitere Kante, wechselt der Algorithmus zurück in Phase 1. Gibt es keine unentdeckte Kante, setzt DFS erneut zurück.
\end{itemize}
Das Verfahren der Tiefensuche endet sobald der Algorithmus zum Startknoten $s$ zurückgesetzt hat und bei $s$ keine unentdeckten Kanten verbleiben. Beim Endecken eines Knoten wird die Distanz zu $s$ berechnet, und der kürzeste Pfad zu $s$ in einen „Tiefensuchbaum“ eingepflegt (vergleiche Breitensuchbaum). Bei der Tiefensuche können jedoch mehrere Bäume entstehen, welche dann einen sogenannten „Tiefensuchwald“ formen \cite{Cormen.2009}. 
Für das Zwischenspeichern der gefundenen Knoten bietet sich ein Stack an, da der Algorithmus selbst nach dem „last in, first out“ Prinzip funktioniert \cite{Tarjan.1972}. Die Tiefensuche hat eine Laufzeit von $\Theta(V + E)$ \cite{Cormen.2009}.


\subsubsection{Anwendung der Tiefensuche}
Neben der Breitensuche wird die Tiefensuche wegen ihres blinden Prinzips selten für Pfadplanung angewendet. DFS kann Irrgärten generieren und einen Weg daraus finden\cite{examMaze}. Darüber hinaus kann DFS für Pfadfindung in einem Gitter verwendet werden\cite{compare}. %todo mark in pdf the text


\subsection{Dijkstras Algorithmus}
In seiner Arbeit „Two Problems in Connexion with Graphs“ stellt Dijkstra zwei Probleme dar \cite{Dijkstra.1959}. Das erste Problem ist die Konstruktion eines Baums mit den kürzesten Wegen zwischen den Knoten eines Graphen (vergleiche Breitensuchbaum). Das andere ist die Suche nach dem kürzesten Weg zwischen zwei Knoten (vergleiche Problem des kürzesten Pfads).
Zur Lösung des Problems des kürzesten Wegs zwischen zwei Knoten $s$ und $d$ wird eine Menge Knoten $R = \{r_1, r_2, ...\}$ definiert, wobei die Knoten R auf dem kürzesten Pfad zwischen $s$ und $d$ liegen. %$R \in min(s \rightarrow d)$ ?
Da ein Knoten $r \in R$ Teil des kürzesten Wegs zwischen $s$ und $d$ ist, ist auch der Weg zwischen $s$ und $r$ minimal. Es werden kontinuierlich Knoten aus $R$ gesucht, die weiter von $s$ entfernt sind, bis der Zielknoten $d$ erreicht wurde.

Zur schrittweisen Lösung nutzt der Algorithmus drei verschiedene Mengen von Knoten. Menge $A = \{a_1, a_2, ...\}$ sind die Knoten, für die die minimale Distanz vom Startknoten $s$ aus bekannt ist. Menge $B = \{b_1, b_2, ...\}$ sind jene Knoten, die direkt mit einem Knoten aus $A$ verbunden sind, aber nicht Teil von $A$ sind. In $C = \{c_1, c_2, ...\}$ befinden sich alle übrigen Knoten.
Zusätzlich wird eine Baumstruktur angelegt, welche das erste von Dijkstra beschriebene Problem löst. Der Baum besitzt drei Hauptzweige. In den ersten beiden Zweigen $Alpha = \{\alpha_1, \alpha_2, ...\}$ und $Beta = \{\beta_1, \beta_2, ...\}$ befinden sich jeweils die Verbindungen vom Startknoten $s$ zu den Knoten der Mengen $A$ und $B$. Pro Knoten der Menge $B$ wird jedoch nur eine Verbindung zu $s$ gespeichert. Im dritten Zweig $Gamma = \{\gamma_1, \gamma_2, ...\}$ befinden sich alle übrigen Verbindungen. 

Zu Beginn des Lösungsprozesses liegen alle Knoten in $C$. Der Startknoten $s$ wird in $A$ abgelegt. Nun werden wiederholend folgende zwei Schritte ausgeführt:
\begin{itemize}
	\item[1.] Es werden alle Verbindungen $H = \{h_1, h_2, ...\}$ zwischen dem zuletzt in $A$ abgelegten Knoten $v$ und den Knoten $R$ aus $B$ oder $C$ überprüft. Liegt der Knoten $r$ in $B$, so wird überprüft, ob die Verbindung $h = \left(s, v, r\right)$ kürzer ist als die im zweiten Zweig befindliche $\left(s, r\right)$. Sollte dies der Fall sein, ersetzt $h$ den im zweiten Zweig abgelegten Pfad zwischen $s$ und $r$. Liegt $r$ in $C$, so wird der Knoten in $B$ verschoben und $h$ dem zweiten Zweig hinzugefügt.
	\item[2.] Werden für die Knoten $B$ lediglich Verbindungen des ersten Zweigs und eine Verbindung des zweiten betrachtet, besitzt jeder Knoten aus $B$ eine feste Länge zu $s$. Der Knoten aus $B$ mit der kürzesten Distanz zu $s$ wird nach $A$ verschoben und seine Verbindung im ersten Zweig des Baums abgelegt.
\end{itemize}
Diese Schritte werden so lange wiederholt bis Zielknoten $d$ in $A$ verschoben wurde.

Der Algorithmus hat eine Komplexität von $O(n^2)$ \cite{Madkour.2017}. Deshalb schlugen Fredman und Tarjan 1984 die Nutzung eines Fibonacci Heaps zur Verbesserung des Dijkstra Algorithmus vor \cite{Fredman.1987}. Unter Anwendung eines solche F-Heaps stellt der Dijkstra Algorithmus den asymptotisch schnellsten bekannten Algorithmus zur Lösung des Shortest Path Problems in gerichteten Graphen dar \cite{Schmitz.2019}.  Die Komplexität beträgt $O(n\ log\ n + m)$ \cite{Madkour.2017}. %todo n und m unterscheiden ?



\subsubsection{Anwendung des Dijkstra-Algorithmus }

Der Suchalgorithmus von Dijkstra und dessen Erweiterungen besitzen ein sehr breites Anwendungsspektrum, zum Beispiel in Transportnetzwerken in sicheren und unsicheren Umgebungen \cite{fuzzyDijk} \cite{publicTrans} %todo eine Quelle weg
. Außerdem findet der Algorithmus Verwendung in Satellitennetzwerken \cite{satelite} und in der Entwicklung von Hardware \cite{hardware}.  %todo noch 1-2 Anwendungen, z.B: Maze Runner Paper

\section{Informierte Algorithmen}

Die sogenannte Bestensuche (engl. informed best-first search) 
ist ein verbreiteter Ansatz für Problemlösungen, um die Suchzeit für die Suche nach dem kürzesten Pfad
anhand heuristischer Informationen zu verkürzen. Dabei wird versucht jedem Punkt in einem Graphen einen Wert zuzuweisen. Anhand dieser Wertzuweisung wird die Erkundung in einem Graphen in die Richtung des vielversprechendsten Kandidaten fortgeführt\cite{RinaDechterandJudeaPearl.1983}. 

\subsection{Heuristiken}
Um die Suche nach einer Lösung schneller zu machen werden heuristische Methoden in Verbindung mit unterschiedlichen Metriken angewendet. Dabei werden Schätzwerte angenähert, mit denen die Algorithmen schneller zum Ziel kommen, indem sie nur einen Teil der verfügbaren Pfade untersuchen\cite{RinaDechterandJudeaPearl.1983}. 
\subsubsection{Metriken}
Für die Distanzabschätzung der heuristischen Methoden gibt es verschiedene Ansätze. Verwendet werden hauptsächlich folgende Metriken.

\begin{itemize}
\item[1.] Die Manhattendistanz misst den Abstand zwischen zwei Knotenpunkten in einem Raster indem sie die horizontale und vertikale Distanz aufeinanderaddiert.

\item[2.] Die Diagonaldistanz arbeitet mit der Annahme, dass Agenten sich in die diagonale genauso wie in die horizontale und vertikale Richtung bewegen können. 

\item[3.] Der Chebyshev-Abstand ist eine Variante der Diagonaldistanz. Bei ihr wird angenommen das die Kosten für die diagonale Traversierung der horizontalen und vertikalen entsprechen. 

\item[4.] Es gibt zwei Varianten der euklidischen Distanz. Es gibt die einfache und die quadrierte euklidische Distanz. Die euklidische Distanz berechnet den Abstand zwischen zwei Knoten als Punkte in einem kartesisches Koordinatensystem. Da die Berechnung der Wurzel für die Abstandsberechnung rechenintensiv ist, wird in der alternativen Variante auf die Ziehung der Wurzel verzichtet und der quadrierte euklidische Abstand verwendet\cite{YouSurLuhu}.

\end{itemize}
\subsection{A*} 
In diesem Zusammenhang wird häufig der A* Algorithmus untersucht. Um den kürzesten Pfad mit dem geringsten Aufwand zu finden, profitiert ein Suchalgorithmus von informierten Entscheidungen für die Untersuchung der Knotenpunkte. Untersucht er Knotenpunkte, die offensichtlich nicht zum Ziel führen können, verschwendet er Ressourcen. Untersucht er andererseits Knoten nicht, welche zum kürzesten Pfad beitragen, führt der Algorithmus nicht zum richtigen Ergebnis.

Es sei ein gewichteter Graph $G$ gegeben, der den Startknoten $s$ und eine nichtleere Menge aus Zielknoten $T$ habe. Das Problem des optimalen Pfades besteht nun darin, den Pfad vom Startknoten $s$ zu einem Knoten aus der Zielmenge $T$ mit den geringsten Kosten zu finden. %todo Wiederholung des Shortest Path Problems, was eigentlich am Anfang richtig definiert werden muss und hier einfach verlinkt
Die Kosten bestehen aus den Gewichten, die den Knoten zugewiesen sind\cite{RinaDechterandJudeaPearl.1983}.

A* benutzt eine spezielle Form der Funktion $f$ zusammengesetzt aus der Summe $f(n) = g(n) + h(n)$. Die Funktion $g(n)$ spiegelt die additiven Kosten des momentanen Pfades von dem Startknoten $s$ zu dem jetzigen Knoten $n$ wieder. Die Funktion $h(n)$ liefert eine heuristische Einschätzung, welche die Kosten des Restpfades zum Zielknoten schätzt\cite{RinaDechterandJudeaPearl.1983}. %todo wie schon besprochen, die h und g müssen besser aus dem KOntext unterschieden werden können 

%todo ich finde n als einzeln Knoten nicht gut, weil n z.B. für eingabelänge bei Laufzeit bei Fredrik benutzt wird, ist besser hier vlt. x oder u 

Der Algorithmus A* folgt bestimmten Schritten. 
\begin{itemize}
\item[1.] Der Startknoten $s$ wird als „offen“ markiert und die Funktion $f(n)$ berechnet. Als “offen” werden die Knoten bezeichnet, die von dem Algorithmus besucht und für die Untersuchung freigeschaltet worden sind.
\item[2.] Derjenige offene Knoten $n$ wird ausgewählt, für den die Funktion $f(n)$
den geringsten Wert liefert. Anschließend werden beliebige angrenzende Knoten ausgewählt, aber immer zugunsten jedes Knotens aus der Zielmenge $T$.
\item[3.] Es wird überprüft, ob sich der aktuelle Knoten $n$ in der Zielmenge befindet. Ist das der Fall, wird er als „geschlossen“ markiert und der Algorithmus terminiert. Ein Knoten $n$ gilt als “geschlossen” sobald die Berechnung der Funktion $f(n)$ für ihn durchgeführt wurde.
\item[4.]
Andernfalls wird der Knoten $n$ als „geschlossen“ markiert und es werden alle Nachfolger von $n$ erkundet. Dann wird $f(n)$ für jeden Nachfolger berechnet und jeder Nachfolger als „offen“ markiert, der nicht als „geschlossen“ markiert wurde. Gibt es nun Nachfolgeknoten die als „geschlossen“ markiert sind, für welche die Funktion $f$ einen geringeren Wert liefert, werden diese Knoten als „offen“ markiert. Daraufhin wird Schritt zwei wieder ausgeführt. Solange, bis der Algorithmus terminiert und der kürzeste Pfad durch den Graphen gefunden wurde\cite{HartNilssonandRaphael.1968}.
\end{itemize}
%todo Komplexität ?



\chapter{Anwendung von Pfadplanungsalgorithmen}

Algorithmen zur Pfadplanung finden in verschiedenen Bereichen Anwendung. Reale Pfadplanungsszenarien, wie Navigation und Logistik, sind durch Graphen modellierbar, auf die Suchalgorithmen angewendet werden können \cite{Botea.2011}. In Computerspielen werden Pfadplanungsalgorithmen für die Bewegung computergesteuerter Charaktere genutzt \cite{compare}. Dieses Gebiet birgt die zusätzliche Herausforderung die Bewegungen der Charaktere realistisch zu simulieren. Auch hat der Computer direkten Zugriff auf Umgebungsinformationen, sodass Schätzungen präzise  vorgenommen werden können.


\section{Vergleich von BFS, Dijkstra und A*}

Der A* Algorithmus wird in vielen Videospielen für die Bewegungskontrolle von Nicht-Spieler Charakteren (engl. non-player characters, NPCs) genutzt \cite{Stamford.2014}. Der Grund dafür ist die in vielen Fällen schnellere Berechnung des kürzesten Pfads gegenüber uninformierten Algorithmen. In ihrer Arbeit "Pathfinding Algorithm Efficiency Analysis in 2D Grid" vergleichen Zarembo und Kodors die Laufzeit der Algorithmen BFS, Dijkstra und A*. Tabelle \ref{tab:comp} zeigt eine Gegenüberstellung der Rechenzeiten der Algorithmen bei verschiedenen Rastergrößen. Die Breitensuche benötigt schon bei sehr kleiner Flächengröße deutlich länger als Dijkstra und A*. Je größe die Fläche wird, desto größer wird auch der Unterschied zwischen Dijkstra und A*. Bei einer Rastergröße von 1024x1024 Knoten braucht A* weniger als die Hälfte der Zeit von Dijkstras Algorithmus. Die Breitensuche benötigt hier die 10.000fache Zeit von A*. Das sind knapp vier Stunden, im Gegensatz zu gut einer Sekunde.

\newcolumntype{C}[1]{>{\centering\arraybackslash}p{#1}}
\begin{table}[h]
\centering
\begin{tabular}[h]{|C{4cm}|C{2cm}|C{2cm}|C{2cm}|} \hline
	Grid size (nodes) & BFS & Dijkstra & A* \\ \hline
	64x64 & 150 & 6 & 4 \\ \hline
	128x128 & 2803 & 25 & 16 \\ \hline
	256x256 & 48313 & 120 & 77 \\ \hline
	512x512 & 821598 & 515 & 265 \\ \hline
	1024x1024 & 13962457 & 2362 & 1148 \\ \hline
\end{tabular}
\caption{Rechenzeit in ms, in Anlehnung an \cite{Zarembo.2013}}
\label{tab:comp}
\end{table}

A* überprüft bei der Suche nach dem kürzesten Pfad in vielen Fällen deutlich weniger Knoten als Dijkstra \cite{compare}. In einer Umgebung mit vielen Hindernissen kommt A* schneller zu einem Ergebnis als Dijkstra. Die Speicher- und Recheneffizienz von A* hängt von der genutzten heuristischen Funktion ab \cite{Noori.2015}. Es ist sinnvoll den A* Algorithmus an die Umgebung anzupassen, in der er genutzt werden soll.

\section{Erweiterung von A*}
Das Problem Pfadfindung in modernen Spielen muss in Echtzeit gelöst werden. Außerdem sind die Speicher und CPU-Ressourcen oft begrenzt. A* hat Worst-case Komplexität wie BFS\cite{astar} und die Pfadfindung mit großen Datensätze kann zu schwerwiegenden Leistungsengpässen führen. 


%todo besserer Einleitungssatz
\subsection{Hierarchical Path-Finding A*}
HPA* ist eine Erweiterung von A*, welcher den durch Abstraktion des Suchraums optimiert\cite{hpa}. Es wird eine bessere Laufzeit erreicht, indem das Spielwelt hierarchisch aufgeteilt wird. 

Es wird das Problem der Reise aus eine Spielstadt nach andere betrachtet. Gegeben sei ein detaillierte Karte mit allen Fahrstrecken und deren Längen in den Städten und zwischen den Städte. HPA* arbeitet nicht mit einem geringen Detaillierungsgrad, wie die einzelne interne Stadtpfaden. Das oben beschriebene Problem kann effizienter gelöst werden, indem zuerst den Pfad auf der Stadtebene und erst dann die Zwischenpfade in jeder durchlaufenden Stadt gesucht werden. Die Hierarchie kann auf mehrere Ebene erweitert werden, wodurch diese Lösung für größere Problembereiche skalierbar wird. HPA funktioniert wie folgt:\\

1. In erstem Schritt wird zur Grenze des Bereichs, der den Startort enthält, fahren.

2. In zweitem wird der Pfad von Startbereich zum Zielbereich gesucht. Das Passiert aber auf eine höhere Abstraktionsebene, wo die Suche schneller ist. 

3. In letztem Schritt wird der Pfad vervollständigt, indem man den Bereich, der das Zielstandort, enthält.\\
\begin{sloppypar}
HPA* läuft durchschnittlich 10 mal schneller als der standard A*-Algorithmus \cite{hpa} und wird daher breit in dem Spieleinbereich angewendet \cite{LeDuc.2008}. HPA* ist aber sehr allgemein. ``In der Praxis kann HPA sowohl für statische Welten, wo die Graphenhierarchie besser vorbereitet werden kann, als auch in typischen dynamischen Welten, um häufige Umgebungsänderungen zu bewältigen, erheblich optimiert werden``\cite{dshpa}. Aus diesem Grund existieren unterschiedliche  HPA*-Erweiterungen wie: SHPA und DHPA vorgeschlagen von Kring et.al.\cite{dshpa} oder Path Smoothing und Lazy Edge Weight Computation, die in allgemeinen die Leistung erhöhen\cite{hpaEnch}.
\end{sloppypar}



\section{Multi-Agent Pathfinding}
Bisher wurden Lösungen für die Problemstellung mit einem einzelnen Agenten betrachtet. Ein Szenario in dem mehrere Akteure einen Pfad finden müssen, bei dem keine Kollisionen auftreten dürfen, bezeichnet man als Multi-Agent Pathfinding-Problem (MAPF). Es gibt verschiedene MAPF-Algorithmen, die versuchen mit dieser Problemstellung umzugehen. Darunter gibt es Windowed Hierarchical Cooperative A*, Flow Annotated Replanning und Bounded Multi-Agent A*.

\subsection{Windowed Hierarchical Cooperative A*}
Bei dem Cooperative A*-Algorithmus sucht jeder Agent mit dem A*-Algorithmus in einem dreidimensionalen Graphen, um sein Ziel zu erreichen. Das Ergebnis der Suche teilt er mit den anderen Agenten in einer Reservierungstabelle. Auf diese Weise sind die Agenten in der Lage Kollisionen zu vermeiden. Der Hierarchical Cooperative A*-Algorithmus benutzt eine hierarchische Suche (siehe Oben: 4.2.1 Hierarchical Path-Finding A*). Windowed Hierarchical Cooperative A*(WHCA*) begrenzt die Suchtiefe für jeden Agent zu einem Fenster. Sobald der Teilpfad in dem Fenster gefunden wurde folgt der Agent diesem Pfad und fängt mit der Berechnung des nächsten Teilpfades an. 
\subsection{Flow Annotated Replanning}
Ähnlich WHCA* nimmt Flow Annotation Replanning (FAR) die Pläne anderer Agenten in die Berechnung auf. Statt aber, wenn ein Pfad blockiert ist, einen neuen Pfad zu berechnen, wartet ein Agent im FAR-Algorithmus einfach auf dem aktuellen Knoten darauf, dass sein Pfad wieder frei wird und er ihn für sich reservieren kann. 

\subsection{Bounded Multi-Agent A*}
Bounded Mulit-Agent A* basiert auf Real-time Adaptive A*(RTAA*). RTAA* wird für ein Multiagenten-Setting erweitert. Andere Agenten werden während der Suche als sich bewegende Hindernisse betrachtet. Außerdem haben Agenten die Möglichkeit andere Agenten dazu aufzufordern Zellen frei zu machen. Der aufgeforderte Agent wird zu einer freien benachbarten Zelle wandern und seine reguläre Such von dort aus fortführen\cite{Sigurdson.2019}.

\section{Verwandte Arbeiten} 
Pfadplanungsproblem wird oft ebenfalls in Robotik Kontext untersucht. In \cite{lozano} und \cite{latombe} wird einer Konfigurationsraum beschrieben. Die Idee besteht darin, dass ein Roboter als ein Punkt betractet wird in diesem Raum betrachtet wird. Nach Latombe\cite{latombe}, es existieren folgende Ansätze für Pfadpalnung in einem Konfigurationsraum: Wegkartenverfahren, Zellunterteilungsverfahren und Potentialfelderverfahren.


Bei \textbf{Wegkartenverfahren} wird der Konfigurationsraum in Kurven unterteilt. Einige Versionen davon sind z.B Voronoi-Diagramm\cite{voronoi} und der \linebreak Sichtbarkeitsgraph\cite{visG1}. Das \textbf{Potentialfeldverfahren} bietet einen Ansatz, bei dem der Suchraum von Gradienten überspannt wird. Aus den Gradienten werden dann mögliche Bewegungsrichtungen ersichtlich. Die \textbf{Zellzerlegung}\cite{cd} kann als Hindernisvermeidungsverfahren bezeichnet werden, das den Konfigurationsraum in eine endliche Sammlung nicht überlappender(disjunkter) konvexer Polygone, sogenannte Zellen, zerlegt.
%todo noch überarbeiten

\chapter{Fazit}

In dieser Arbeit wurde die Problemstellung des kürzesten Pfads dargestellt. Es wurden die grundlegenden Funktionsprinzipien von Algorithmen zur Pfadplanung erklärt. Der Unterschied zwischen uninformierten und informierten Algorithmen wurde herausgearbeitet und verschiedene Metriken für Schätzverfahren genannt. Zusätzlich wurden die besonderen Herausforderungen verschiedener Anwendungsgebiete diskutiert und Ansätze der umgebungsspezifischen Optimierung erläutert. Abschließend wurde der Ansatz von Multi-Agent Pathfinding-Algorithmen zur parallelen Suche mehrerer, kollisionsfreier Pfade gegeben.

% ...
%--------------------------------------------------------------------------
\backmatter                        		% Anhang
%-------------------------------------------------------------------------
\bibliographystyle{geralpha}			% Literaturverzeichnis
\bibliography{literatur_vlad,literatur_fredrik,bib_valentin}     			% BibTeX-File literatur.bib
%--------------------------------------------------------------------------
\printindex 							% Index (optional)
%--------------------------------------------------------------------------
\begin{appendix}						% Anh�nge sind i.d.R. optional
   %\chapter{Glossar}

			% Glossar   
\end{appendix}

\end{document}
