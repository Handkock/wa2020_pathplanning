%%%%%%%%%%%%%%%%%%% vorlage.tex %%%%%%%%%%%%%%%%%%%%%%%%%%%%%
%
% LaTeX-Vorlage zur Erstellung von Projekt-Dokumentationen
% im Fachbereich Informatik der Hochschule Trier
%
% Basis: Vorlage svmono des Springer Verlags
%
%%%%%%%%%%%%%%%%%%%%%%%%%%%%%%%%%%%%%%%%%%%%%%%%%%%%%%%%%%%%%

\documentclass[envcountsame,envcountchap, deutsch]{i-studis}

\usepackage{makeidx}         	% Index
\usepackage{multicol}        	% Zweispaltiger Index
%\usepackage[bottom]{footmisc}	% Erzeugung von Fu�noten

%%-----------------------------------------------------
%\newif\ifpdf
%\ifx\pdfoutput\undefined
%\pdffalse
%\else
%\pdfoutput=1
%\pdftrue
%\fi
%%--------------------------------------------------------
%\ifpdf
\usepackage[pdftex]{graphicx}
\usepackage{epstopdf}
\usepackage[pdftex,plainpages=false]{hyperref}
%\else
%\usepackage{graphicx}
%\usepackage[plainpages=false]{hyperref}
%\fi

%%-----------------------------------------------------
\usepackage{color}				% Farbverwaltung
%\usepackage{ngerman} 			% Neue deutsche Rechtsschreibung
\usepackage[english, ngerman]{babel}
\usepackage[latin1]{inputenc} 	% Erm�glicht Umlaute-Darstellung
%\usepackage[utf8]{inputenc}  	% Erm�glicht Umlaute-Darstellung unter Linux (je nach verwendetem Format)

%-----------------------------------------------------
\usepackage{listings} 			% Code-Darstellung
\lstset
{
	basicstyle=\scriptsize, 	% print whole listing small
	keywordstyle=\color{blue}\bfseries,
								% underlined bold black keywords
	identifierstyle=, 			% nothing happens
	commentstyle=\color{red}, 	% white comments
	stringstyle=\ttfamily, 		% typewriter type for strings
	showstringspaces=false, 	% no special string spaces
	framexleftmargin=7mm, 
	tabsize=3,
	showtabs=false,
	frame=single, 
	rulesepcolor=\color{blue},
	numbers=left,
	linewidth=146mm,
	xleftmargin=8mm
}
\usepackage{textcomp} 			% Celsius-Darstellung
\usepackage{amssymb,amsfonts,amstext,amsmath}	% Mathematische Symbole
\usepackage[german, ruled, vlined]{algorithm2e}
\usepackage[a4paper]{geometry} % Andere Formatierung
\usepackage{bibgerm}
\usepackage{array}
\hyphenation{Ele-men-tar-ob-jek-te  ab-ge-tas-tet Aus-wer-tung House-holder-Matrix Le-ast-Squa-res-Al-go-ri-th-men} 		% Weitere Silbentrennung bei Bedarf angeben
\setlength{\textheight}{1.1\textheight}
\pagestyle{myheadings} 			% Erzeugt selbstdefinierte Kopfzeile
\makeindex 						% Index-Erstellung


%--------------------------------------------------------------------------
\begin{document}
%------------------------- Titelblatt -------------------------------------
\title{Funktionsprinzipien und Anwendungen von Algorithmen zur Pfadplanung}
\project{Ausarbeitung zur Vorlesung Wissenschaftliches Arbeiten}
%--------------------------------------------------------------------------
\supervisor{Titel Vorname Name} 		% Betreuer der Arbeit
\author{Bearbeiter 1: Fredrik Kappmeier \\Bearbeiter 2: Vladislav Paskar \\Bearbeiter 3: Valentin Thum}							% Autor der Arbeit
\groupid{146}
\address{Trier,} 							% Im Zusammenhang mit dem Datum wird hinter dem Ort ein Komma angegeben
\submitdate{3.07.2020} 				% Abgabedatum
%\begingroup
%  \renewcommand{\thepage}{title}
%  \mytitlepage
%  \newpage
%\endgroup
\begingroup
  \renewcommand{\thepage}{Titel}
  \mytitlepage
  \newpage
\endgroup
%--------------------------------------------------------------------------
\frontmatter 
%--------------------------------------------------------------------------
\kurzfassung

Die folgende Arbeit beschäftigt sich mit der Pfadplanung. Diese hat das Ziel %todo komma ?
den kürzesten Pfad zwischen Start- und Endpunkt zu finden. Die Breiten- und Tiefensuche sind grundlegende Lösungsansätze, die eine optimale Lösung garantieren. Sie gehören zu den uninformierten Algorithmen, die blind arbeiten. Informierte Algorithmen arbeiten mit heuristischen Informationen, um die Suche zu verkürzen. Dijkstras Algorithmus ist einer der wichtigsten Algorithmen in der Pfadplanung. Er stellt die Grundlage für den in der Anwendung weit verbreiteten A* Algorithmus dar. Zur Anpassung an spezifische Umgebungseigenschaften existieren diverse Erweiterungen von A*, wie hierarchisches A* Pathfinding. Multi-Agent Pathfinding-Algorithmen planen für mehrere Akteure kollisionsfreie kürzeste Pfade.
 			% Kurzfassung Deutsch/English
\tableofcontents 						% Inhaltsverzeichnis
%--------------------------------------------------------------------------
\mainmatter                        		% Hauptteil (ab hier arab. Seitenzahlen)
%--------------------------------------------------------------------------
% Die Kapitel werden in separaten .tex-Dateien abgelegt und hier eingebunden.
\chapter{Einleitung und Problemstellung}

Die Pfadplanung ist ein fundamentales Problem in vielen Bereichen wie Computerspiele \cite{}, Robotik \cite{}, Navigation und Logistik \cite{Botea.2011}. Statische und dynamische Umgebungen stellen unterschiedliche Herausforderungen dar. Weiter gibt es Szenarien, in denen mehrere Akteure gleichzeitig Pfade finden müssen. Um diese Probleme anzugehen, gibt es verschiedene Ansätze, die in dieser Arbeit diskutiert werden. Dafür existieren unterschiedliche Kategorien von Algorithmen. Diese Arbeit definiert die Problemstellung, gibt eine Einführung in die Algorithmen zur Lösung des Problems und führt verschiedene Anwendungsszenarien auf. Es wird grundlegendes Verständnis für die Funktionsprinzipien fundamentaler Pfadplanungsalgorithmen aufgebaut und Ansätze für die umgebungsbedingte Optimierung erläutert. Dazu wird die Effizienz der verschiedenen Pfadplanungsprinzipien beleuchtet.
\chapter{Arbeitsbereich Frederik}

\chapter{Arbeitsbereich Valentin}

\chapter{Arbeitsbereich Vladislav}

\chapter{Fazit}

In dieser Arbeit wurde die Problemstellung des kürzesten Pfads dargestellt. Es wurden die grundlegenden Funktionsprinzipien von Algorithmen zur Pfadplanung erklärt. Der Unterschied zwischen uninformierten und informierten Algorithmen wurde herausgearbeitet und verschiedene Metriken für Schätzverfahren genannt. Zusätzlich wurden die besonderen Herausforderungen verschiedener Anwendungsgebiete diskutiert und Ansätze der umgebungsspezifischen Optimierung erläutert. Abschließend wurde der Ansatz von Multi-Agent Pathfinding-Algorithmen zur parallelen Suche mehrerer, kollisionsfreier Pfade gegeben.

% ...
%--------------------------------------------------------------------------
\backmatter                        		% Anhang
%-------------------------------------------------------------------------
\bibliographystyle{geralpha}			% Literaturverzeichnis
\bibliography{literatur}     			% BibTeX-File literatur.bib
%--------------------------------------------------------------------------
\printindex 							% Index (optional)
%--------------------------------------------------------------------------
\begin{appendix}						% Anh�nge sind i.d.R. optional
   \chapter{Glossar}

			% Glossar   
\end{appendix}

\end{document}
